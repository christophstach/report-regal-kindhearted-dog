\documentclass[chapterprefix=false, 12pt, a4paper, oneside, parskip=half, listof=totoc, bibliography=totoc, numbers=noendperiod]{scrbook}

\usepackage[utf8]{inputenc}
\usepackage[T1]{fontenc}
\usepackage[bottom=48mm,left=25mm,right=25mm]{geometry}
\usepackage[onehalfspacing]{setspace}
\usepackage[stretch=10]{microtype}
\usepackage{scrhack}
\usepackage{listings}
\usepackage[ngerman]{babel}

\renewcommand*{\chapterheadstartvskip}{\vspace*{.25\baselineskip}}


\title{Bericht über Werkstudententätig bei der DERICON GmbH}
\author{Christoph Stach}
\date{01.10.2016 - 31.09.2018}


\begin{document}
    \begin{titlepage}
        \maketitle
    \end{titlepage}

    \tableofcontents

    \chapter{Allgemeines}

    Dieser Bericht behandelt die Werkstudententätigkeit von Christoph Stach bei der DERICON GmbH im Zeitraum vom 01.10.2016 bis 31.09.2018.

    \section{Beschreibung des Unternehmens}

    Die DERICON GmbH ist ein bankenunabhängiges Finanzdienstleistungsinstitut mit Standorten in Frankfurt am Main und Berlin.
    Das Unternehmen unterstützt seit 2008 Banken und Vermögensverwalter bei der Gestaltung effizienter und rechtskonformer Beratungs- und Vertriebsprozesse.
    Mit DERIFIN betreibt DERICON dafür die führende Webanwendung zur Selektion, Steuerung und Risikomanagement von strukturierten Produkten.
    Zudem liefert DERICON professionelle Daten und Kennzahlen für die Analyse und den Einsatz strukturierter Produkte.
    Europaweit vertrauen bereits mehr als 60 Privatbanken, Sparkassen und Genossenschaftsinstitute auf die Expertise von DERICON.
    \\ \\
    Im Jahr 2015 wurde über DERIFIN ein Anlagevolumen von rund 1,2 Mrd. Euro vermittelt.

    \section{Beschreibung der Produkte}

    Das Hauptprodukt von DERICON ist die Webanwendung DERIFIN. Außerdem wurde während ich bei der DERICON GmbH eingestellt war,
    eine neue Version von DERIFIN entwickelt, das sogennante DERIFIN WMS. Welches mehr Features bietet und mit neueren Technologien entwickelt wurde.
    Neben der Hauptsoftware DERIFIN entwickelt DERICON viele internete Produkte zur besseren Verwaltung von DERIFIN und seiner Kunden.

    \section{Beschreibung des Teams}

    Das Team von DERICON ist auf zwei Standorte verteilt. Der erste Standort ist in Frankfurt am Main und der zweite Standort in Berlin Charlottenburg.
    Mein Arbeitsplatz ist im Berliner Büro. Zum heutigen Zeitpunkt arbeiten zwei Frontend-Entwickler, sowie einer der Geschäftsführer in diesem Büro.
    Im Frankfurter Büro arbeiten überwiegend Backend-Entwickler sowie ein weiterer Geschäftsführer und die Buchhaltung.
    \\ \\
    Die komplette Entwicklungsabteilung hält jeden Morgen ein Meeting per Videotelefonie, in dem jedes Mitglied kurz berichtet
    was er am Vortag gemacht hat und was er an diesem Tag vorhat.
    Die Entwicklung ist in Sprints von zwei Wochen gegegliedert. Deswegen gibt es ein weiteres Sprintplanungsmeeting am Anfang eines jeden Sprints.
    Dieser Prozess ist an SCRUM angelehnt.

    \section{Organisation}

    \chapter{Aufgaben}

    \section{CitrusNG}

    \subsection{Ausgangsposition}

    \subsection{Entwicklung}

    \subsection{Ergebnisse}

    \section{BrokerUI}

    \subsection{Ausgangsposition}

    \subsection{Entwicklung}

    \subsection{Ergebnisse}

    \section{DERIFIN WMS}

    \subsection{Ausgangsposition}

    \subsection{Entwicklung}

    \subsection{Ergebnisse}

    \chapter{Schlussfolgerungen}

    \section{Zusammenhänge mit dem Studium an der HTW Berlin}

\end{document}